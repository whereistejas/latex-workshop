\documentclass[]{beamer}

% beamer options
	%\setbeamercovered{transparent}
% custom commands
	\def\colorize<#1>{%
\temporal<#1>{\color{red!50}}{\color{black}}{\color{black!50}}}

\usepackage{tikz}
\usetikzlibrary{shapes.geometric, arrows}

\author{Tejas Sanap}
\title{Introduction to \LaTeX}
\date{December 28, 2019}

\begin{document}
	\begin{frame}
		\titlepage
	\end{frame}

	\begin{frame}
		\frametitle{A few words about myself}
		\begin{itemize}
			\item I mainly work in IDAM.
			\item I work at Wipro.
			\item I am an organizer at (and, member of ) PLUG.
			\item I do a lot of open-source stuff.
		\end{itemize}
	\end{frame}

	\begin{frame}
		\frametitle{Type-setting?}
		\onslide<1->{What is type-setting?}

		\onslide<2->{
		\begin{itemize}
			\item The process of arranging the various objects on a page.
			\item It is process that takes place after the manuscript has been written.
		\end{itemize}
		}
		\onslide<3->{How do we typeset? \\}
		\onslide<4->{
		\vfill
		\huge{MS WORD!}
		}
	\end{frame}

	\begin{frame}
		\frametitle{Why do I need \LaTeX?}
		\onslide<1->{ To save time and efforts. }
		\begin{itemize}
			\item<2-> \LaTeX does all the type-setting for you.
			\item<3-> It also auto-generates:
				\begin{itemize}
					\colorize<3> \item Table of content
					\colorize<4> \item List of figures and tables
					\colorize<5> \item Captions
					\colorize<6> \item Headers and footers
					\colorize<7> \item Page numbers (both roman and decimal)
				\end{itemize}
		\end{itemize}
	\end{frame}

	\begin{frame}
		\frametitle{What is \LaTeX?}
		\begin{itemize}
			\item A type-setting system.
			\item An improvement over \TeX.
			\item Not a WYSIWYG editor.
			\item Uses simple plaintext.
		\end{itemize}
	\end{frame}

	\begin{frame}
		\frametitle{How do I use \LaTeX?}
		\tikzstyle{block} = [rectangle, draw=red, dashed, thick]
		\tikzstyle{arrow} = [thick, ->, >=stealth]
		\tikzstyle{arrow1} = [thick, ->, >=stealth, draw=red]
		\begin{tikzpicture}{node=2cm}
			\onslide<1>{\node (texfile) [] {\texttt{.tex}};}
			\onslide<2>{\node (texfile) [block] {\texttt{.tex}};}

			\node (pdfengine) [right of=texfile, xshift=3cm] {\texttt{xelatex}};
			\onslide<1>{\node (xeengine) [above of=pdfengine, yshift=2cm] {\texttt{pdftex}};}
			\onslide<2>{\node (xeengine) [block, above of=pdfengine, yshift=2cm] {\texttt{pdftex}};}
			\node (luaengine) [below of=pdfengine, yshift=-2cm] {\texttt{lualatex}};
			\onslide<1>{\node (pdffile) [right of=pdfengine, xshift=3cm] {\texttt{.pdf}};}
			\onslide<2>{\node (pdffile) [block, right of=pdfengine, xshift=3cm] {\texttt{.pdf}};}
			\node (psfile) [above of=pdffile, yshift=2cm] {\texttt{.ps}};
			\node (dvifile) [below of=pdffile, yshift=-2cm] {\texttt{.dvi}};

			\node (source) [below of=texfile, yshift=-3cm] {\textbf{Source Code}};
			\node (source) [below of=pdfengine, yshift=-3cm] {\LaTeX\,\textbf{engine}};
			\node (source) [below of=pdffile, yshift=-3cm] {\textbf{Final Output}};

			\draw [arrow] (texfile) -- (pdfengine);
			\onslide<1>{\draw [arrow] (texfile) -- (xeengine);}
			\onslide<2>{\draw [arrow1] (texfile) -- (xeengine);}
			\draw [arrow] (texfile) -- (luaengine);
			\draw [arrow] (pdfengine) -- (pdffile);
			\draw [arrow] (pdfengine) -- (psfile);
			\draw [arrow] (pdfengine) -- (dvifile);
			\onslide<1>{\draw [arrow] (xeengine) -- (pdffile);}
			\onslide<2>{\draw [arrow1] (xeengine) -- (pdffile);}
			\draw [arrow] (xeengine) -- (psfile);
			\draw [arrow] (xeengine) -- (dvifile);
			\draw [arrow] (luaengine) -- (pdffile);
			\draw [arrow] (luaengine) -- (psfile);
			\draw [arrow] (luaengine) -- (dvifile);
		\end{tikzpicture}
	\end{frame}

\end{document}
