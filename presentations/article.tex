\documentclass{beamer}

\usepackage{minted}
\usepackage{tikz}
\usepackage{fancyvrb}

\usetikzlibrary{shapes.geometric, arrows}

\title{How to write an \texttt{article}}
\author{Tejas Sanap}
\date{\today}

\begin{document}
	\begin{frame}
		\titlepage
	\end{frame}

	\begin{frame}
		\frametitle{Outline}
		\tableofcontents
	\end{frame}

	\section{Document Structure}
	\begin{frame}[fragile]
		\frametitle{Sectioning}
		% define block styles
		\tikzstyle{box}=[rectangle, draw=black, thick, minimum width=2.5cm]
		\tikzstyle{arrow}=[draw, thick, ->, >=stealth]
			
		\begin{onlyenv}<1>
			\begin{center}
				\begin{tikzpicture}[node distance=1cm]
					%place nodes
					\node [box] (chap) {\verb+\chapter{}+};
					\node [box, below of=chap] (sec) {\verb+\section{}+};
					\node [box, below of=sec] (ssec) {\verb+\subsection{}+};
					\node [box, below of=ssec] (sssec) {\verb+\subsubsection{}+};
					\node [box, below of=sssec] (para) {\verb+\paragraph{}+};
					\node [box, below of=para] (spara) {\verb+\subparagraph{}+};
					%join edges
					\path [arrow] (chap) -- (sec);
					\path [arrow] (sec) -- (ssec);
					\path [arrow] (ssec) -- (sssec);
					\path [arrow] (sssec) -- (para);
					\path [arrow] (para) -- (spara);
				\end{tikzpicture}
			\end{center}
		\end{onlyenv}

		\begin{onlyenv}<2>
			Sectional commands have the following two forms:
			\begin{description}
				\item[\mintinline{latex}{\section{}}] Numbered sections.
				\item[\mintinline{latex}{\section*{}}] Numbered sections.
			\end{description}
		\end{onlyenv}
	\end{frame}

	\section{Lists}
	\begin{frame}[fragile]
		\frametitle{Types of Lists}
		\only<1>{
			There are three types of lists in \LaTeX:
			\begin{itemize}
				\item \texttt{itemize}.
				\item \texttt{enumerate}.
				\item \texttt{description}.
			\end{itemize}
		}
		\begin{onlyenv}<2>
			Syntax:
			\begin{minted}[autogobble]{latex}
				\begin{itemize}
					\item Naruto Uzumaki.
					\item Sakura
					\item Sasuke Uchiha.
				\end{itemize}
			\end{minted}
		\end{onlyenv}
	\end{frame}
	
	\section{Links}
	\begin{frame}
		\frametitle{Links}
		Various \texttt{hyperref} coloring options are:
		\begin{description}[align=left]
			\item[\mintinline{latex}{linkcolor}]
			\item[\mintinline{latex}{urlcolor}]
			\item[\mintinline{latex}{citecolor}]
			\item[\mintinline{latex}{filecolor}]
			\item[\mintinline{latex}{linkbordercolor}]
			\item[\mintinline{latex}{urlbordercolor}]
			\item[\mintinline{latex}{citebordercolor}]
		\end{description}
	\end{frame}

	\section{Tables}
	\begin{frame}
		\frametitle{General rules to typeset tables}
			\begin{enumerate}
				\item Never use vertical lines.
				\item Avoid double lines.
				\item Place the units in the headings.
				\item Do not use quotation marks to repeat the content of the cells.
			\end{enumerate}
	\end{frame}
\end{document}
