\documentclass{IEEEtran}

\usepackage[inline]{enumitem}

\title{Some supercool article}
\author{Tejas Sanap}

\begin{document}
	\maketitle

	\section{Introduction}
		The Big Bang theory is a cosmological model for the observable universe from the earliest known periods through its subsequent large-scale evolution\cite{2004biba.book.....S}. The model describes how the universe expanded from a very high-density and high-temperature state, and offers a comprehensive explanation for a broad range of phenomena, including the abundance of light elements, the cosmic microwave background (CMB), large-scale structure and Hubble's law (the farther away galaxies are, the faster they are moving away from Earth). If the observed conditions are extrapolated backwards in time using the known laws of physics, the prediction is that just before a period of very high density there was a singularity which is typically associated with the Big Bang. Current knowledge is insufficient to determine if the singularity was primordial.
		
		Georges Lemaître first noted in 1927 that an expanding universe could be traced back in time to an originating single point, calling his theory that of the "primeval atom". The scientific community was once divided between supporters of two different theories, the Big Bang and the steady state theory, but a wide range of empirical evidence has strongly favored the Big Bang which is now universally accepted\cite{kragh1999cosmology}. In 1929, from analysis of galactic redshifts, Edwin Hubble concluded that galaxies are drifting apart; this is important observational evidence for an expanding universe. In 1964, the cosmic microwave background radiation was discovered, which was crucial evidence in favor of the hot Big Bang model\cite{partridge20073k}, since that theory predicted the existence of background radiation throughout the universe before it was discovered.

\end{document}
