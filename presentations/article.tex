\documentclass{beamer}

\usepackage{minted}
\usepackage{tikz}
\usepackage{fancyvrb}
\usepackage{booktabs}
\usepackage[font=scriptsize]{caption}

\usetikzlibrary{shapes.geometric, arrows}

\newcommand{\mysc}[1]{\textrm{\textsc{#1}}}
\newcommand{\mylf}[1]{\textrm{\textlf{#1}}}

\setbeamertemplate{footline}[frame number]

\title{How to write an \texttt{article}}
\subtitle{\tiny Clue: Make a really cool bibliography}
\author{Tejas Sanap}
\date{\today}

\begin{document}
	\begin{frame}
		\titlepage
	\end{frame}

	\begin{frame}
		\frametitle{Outline}
		\tableofcontents
	\end{frame}

	\section{Document Structure}
	\begin{frame}[fragile]
		\frametitle{Sectioning}
		% define block styles
		\tikzstyle{box}=[rectangle, draw=black, thick, minimum width=2.5cm]
		\tikzstyle{arrow}=[draw, thick, ->, >=stealth]
			
		\begin{onlyenv}<1>
			\begin{center}
				\begin{tikzpicture}[node distance=1cm]
					%place nodes
					\node [box] (chap) {\verb+\chapter{}+};
					\node [box, below of=chap] (sec) {\verb+\section{}+};
					\node [box, below of=sec] (ssec) {\verb+\subsection{}+};
					\node [box, below of=ssec] (sssec) {\verb+\subsubsection{}+};
					\node [box, below of=sssec] (para) {\verb+\paragraph{}+};
					\node [box, below of=para] (spara) {\verb+\subparagraph{}+};
					
					\path [arrow] (chap) -- (sec);
					\path [arrow] (sec) -- (ssec);
					\path [arrow] (ssec) -- (sssec);
					\path [arrow] (sssec) -- (para);
					\path [arrow] (para) -- (spara);
				\end{tikzpicture}
			\end{center}
		\end{onlyenv}

		\begin{onlyenv}<2>
			Sectional commands have the following two forms:
			\begin{description}
				\item[\mintinline{latex}{\section{}}] Numbered sections.
				\item[\mintinline{latex}{\section*{}}] Numbered sections.
			\end{description}
		\end{onlyenv}
	\end{frame}

	\section{Font typefaces and sizes}
	\begin{frame}[t, fragile]
		\frametitle{Typefaces}
			\vfill
			There are 13 font families:
			\vfill
			\begin{description}[align=left, margin=0.5cm]
				\item<+->[\mintinline{latex}{\textnormal}]  \textnormal{Document font family}.
				\item<+->[\mintinline{latex}{\emph}] 		\emph{emphasis, typically italics}.
				\item<+->[\mintinline{latex}{\textrm}] 		\textrm{Roman font family}.
				\item<+->[\mintinline{latex}{\textsf}] 		\textsf{San serif font family}.
				\item<+->[\mintinline{latex}{\texttt}] 		\texttt{Teletypefont or monospace font}.
				\item<+->[\mintinline{latex}{\textup}] 		\textup{Upright shape, same as normal font}.
				\item<+->[\mintinline{latex}{\textit}] 		\textit{Italic shape}.
				\item<+->[\mintinline{latex}{\textsl}] 		\textsl{Slanted shape}.
				\item<+->[\mintinline{latex}{\textsc}] 		\mysc{Small capitals}.
				\item<+->[\mintinline{latex}{\uppercase}] 	\uppercase{Uppercase, all caps}.
				\item<+->[\mintinline{latex}{\textbf}] 		\textbf{Bold font}.
				\item<+->[\mintinline{latex}{\textmd}] 		\textmd{medium weight}.
				% \item[\mintinline{latex}{\textlf}] 		\mylf{light}.
			\end{description}
			\vfill
	\end{frame}

	\begin{frame}[t]
		\frametitle{Sizes}
			\vfill
			There are 9 font sizes:
			\vfill
			\begin{description}[align=left, margin=0.5cm]
				\item<+->[\tiny \mintinline{latex}{\tiny}] \tiny{tiny}.
				\item<+->[\mintinline{latex}{\scriptsize}] \scriptsize{sciprtsize}.
				\item<+->[\mintinline{latex}{\footnotesize}] \footnotesize{footnotesize}.
				\item<+->[\mintinline{latex}{\small}] \small{small}.
				\item<+->[\mintinline{latex}{\normalsize}] \normalsize{normalsize}.
				\item<+->[\mintinline{latex}{\large}] \large{large}.
				\item<+->[\mintinline{latex}{\Large}] \Large{Large}.
				\item<+->[\mintinline{latex}{\huge}] \huge{huge}.
				\item<+->[\mintinline{latex}{\Huge}] \Huge{Huge}.
			\end{description}
			\vfill
	\end{frame}

	\section{Lists}
	\begin{frame}[fragile]
		\frametitle{Types of Lists}
		\only<1>{
			There are three types of lists in \LaTeX:
			\begin{itemize}
				\item \texttt{itemize}.
				\item \texttt{enumerate}.
				\item \texttt{description}.
			\end{itemize}
		}
		\begin{onlyenv}<2>
			Syntax:
			\begin{minted}[autogobble]{latex}
				\begin{itemize}[align=left]
					\item Naruto Uzumaki.
					\item Sakura
					\item Sasuke Uchiha.
				\end{itemize}
			\end{minted}
		\end{onlyenv}
	\end{frame}

	\begin{frame}[t]
		\frametitle{How do I modify lists?}
			\vfill
			Use a package called \texttt{enumitem}.
			\vfill
			It lets us change:
			\begin{itemize}
				\item Labels.
				\item Margins.
				\item Alignment.
				\item Direction (vertical or inline).
			\end{itemize}
			\vfill
	\end{frame}
	
	\section{Links}
	\begin{frame}
		\frametitle{How do I insert links?}
			\only<1-2>{
				Use the command	\mintinline{latex}{\url} or \mintinline{latex}{\href}
				\vskip 1\baselineskip
				\onslide<2>{We can refer to internal objects by: \mintinline{latex}{\href{item-label}{text}} or \mintinline{latex}{\ref}}.
			}
	\end{frame}

	\begin{frame}
		\frametitle{How do I change link colors?}
			\vfill
			Use a package called \texttt{hyperref}.
			\vfill
			It lets us:
			\begin{itemize}
				\item Change link colors.
				\item Add bookmarks.
				\item Set PDF properties.
				\item Cross-referencing.
			\end{itemize}
			\vfill
	\end{frame}

	\begin{frame}
		\frametitle{Types of Links}
			\only<1>{
				The four types of links are:
				\begin{itemize}
					\item \mintinline{latex}{link}
					\item \mintinline{latex}{url} 
					\item \mintinline{latex}{cite}
					\item \mintinline{latex}{file}
				\end{itemize}
			}
			\only<2>{
				The various link colors are:
				\begin{itemize}
					\item \mintinline{latex}{linkcolor}
					\item \mintinline{latex}{urlcolor}
					\item \mintinline{latex}{citecolor}
					\item \mintinline{latex}{filecolor}
					\item \mintinline{latex}{linkbordercolor}
					\item \mintinline{latex}{urlbordercolor}
					\item \mintinline{latex}{citebordercolor}
				\end{itemize}
			}
	\end{frame}

	\section{Tables}
	\begin{frame}
		\frametitle{General rules to typeset tables}
			\begin{enumerate}
				\item Never use vertical lines.
				\item Avoid double lines.
				\item Place the units in the headings.
				\item Don't use quotation marks to repeat content.
			\end{enumerate}
	\end{frame}

	\begin{frame}[fragile]
		\frametitle{How to make a table?}
		\begin{onlyenv}<1-2>
			To make a table we use the \texttt{tabular} environment.
			\vfill
			\begin{visibleenv}<2>
				Syntax:
				\begin{minted}[beameroverlays, linenos, autogobble]{latex}
					\begin{tabular}{c c c}
						1 & 2 & 3 \\
						1 & 2 & 3 \\
						1 & 2 & 3 \\
					\end{tabular}
				\end{minted}
			\end{visibleenv}
			\vfill
		\end{onlyenv}
		\only<3->{
			\begin{itemize}
				\item<4-> Tables are floats.
				\item<5-> They need to be put inside the \texttt{table} environment.
				\item<6-> Columns can be aligned as:
					\begin{description}
						\item<7->[Horizontal alignment] \texttt{l, c, r}.
						\item<8->[Vertical alignment] \texttt{p, m, b}.
					\end{description}
				\item<9-> Vertical alignment needs \texttt{array} package.
			\end{itemize}
		}
	\end{frame}

	\begin{frame}
		\frametitle{How to make my tables prettier?}
		\only<1>{
		\vskip1\baselineskip
		\vskip1\baselineskip
		\begin{table}[h]
			\begin{tabular}{|l|c|c|} \hline
					Name & Subject & Marks \\\hline
					Naruto & Taijutsu & 80 \\\hline
					Shiba & Ninjutsu & 82 \\\hline
					Sakura & Genjutsu & 99 \\\hline
					Boruto & All jutsu & 00 \\\hline
			\end{tabular}
			\caption{Marks of various students in Ninja Academy.}
			\label{table:normal-ninja-marks}
		\end{table}
		}
		\only<2-3>{
		\onslide<2->{Use a package called \texttt{booktabs}.}
		\begin{table}[h]
			\begin{tabular}{lcc}
				\toprule
					Name & Subject & Marks \\
				\midrule
					Naruto & Taijutsu & 80 \\
					Shiba & Ninjutsu & 82 \\
					Sakura & Genjutsu & 99 \\
					Boruto & All jutsu & 00 \\
				\bottomrule
			\end{tabular}
			\caption{Marks of various students in Ninja Academy.}
			\label{table:booktabs-ninja-marks}
		\end{table}
		}
	\end{frame}

\end{document}
