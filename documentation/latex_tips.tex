\documentclass[12pt]{article}

\textwidth166mm
\textheight232mm
\oddsidemargin00mm
\topmargin-13mm

\usepackage{graphicx}
\usepackage{color}
\usepackage{bm}
\usepackage{amssymb}
\usepackage{epsfig}
\usepackage{wrapfig}
\usepackage{subfigure}



\begin{document}


\centerline{\bf \huge \LaTeX Tips: How to Beautify Equations}

\smallskip

\centerline{\footnotesize{for PHYS460 and PHYS660 courses at the University of Delaware}}

\bigskip

\textcolor{red}{\em \underline{How to use this file:} Compile the source of this file to get the dvi file, 
and then for each equation you find visually appealing or suggestion about other elements of an article (such as labeling of equations and figures, quotation marks, citation of references, \dots) please take a look at the ASCII source to see how particular trick should be typed.} 

\bigskip

There are many ways to type equations in \LaTeX, which although yielding mathemetically correct equations, 
are not esthetically pleasing enough. For example, Leslie Lamport (the inventor of \LaTeX macros which have greatly simplified plain \TeX system invented by Donald Knuth)  gives the following examples  of using small ``positive'' and ``negative'' spaces on page 51 of Ref.~\cite{lamport} (take a look at the source to see how to properly type here quotation marks which should ``curl'' in different directions at the beginning and at the end of a word).  Instead of $\sqrt{2}x$ we should type $\sqrt{2} \, x$; instead of $n / \log n$ use $n / \! \log n$; instead of $\int \int z dx dy$ use $\int \! \! \int z \, dx dy$.

At this point it is also advantageous to check how the Reference~\cite{lamport} is cited above, as well as its format (which differ from journal to journal, but this style used here is required by Physical Review). If you want to cite the course textbook, please take a look at Ref.~\cite{giordano} below. To cite a journal reference of Project 2, shown below as Ref.~\cite{whelan}, you can just copy and paste its source version, as well as to continue using the same style in the future (where you show authors by their last names and first initial, first page of the article, journal volume in bold, etc).

To make equations which have appeared in Project 1 look more elegant (or in accord with the style guides of 
Rev\TeX4, please take a look at following suggestions and tips:

\begin{itemize}

\item While symbols in math formulae usually appear in italic, some of them should remain in roman--that is we should see \textcolor{red}{$\exp(x)$} and \textcolor{red}{$\sin(x)$} instead of \textcolor{blue}{$exp(x)$} and \textcolor{blue}{$sin(x)$},

\item if you need to use a mathematical function which does not have specific command associated with it (as in the above cases of $\exp$ and $\sin$), you can always change style within math environment so that in Project 1 you get ${\rm max}(\tau_A/\tau_B,1)$ rather than $max(\tau_A/\tau_B,1)$,

\item usage of commands for fraction should be correlated with the position of the fraction within the formula---for example, if embedded into the text, it is much nicer to see $\tau_A/\tau_B$ instead of $\frac{\tau_A}{\tau_B}$; similarly, in the displayed equations it is better to have 
\begin{equation} \label{eq:exp_function}
e^{-t/\tau} \hspace {0.5in} {\rm instead \ of} \hspace{0.5in} e^{-\frac{t}{\tau}},
\end{equation}
or
\begin{equation}
\exp \left(-\frac{t}{\tau} \right) \hspace {0.5in} {\rm instead \ of} \hspace{0.5in} \exp(-\frac{t}{\tau}),
\end{equation}
which is also the style adopted in the textbook (obviously typed in some kind of \TeX! package - many publishers have their own set of macros defined on the top of plain \TeX), as you can see in Eq.~(1.2) on page 1 of Ref.~\cite{giordano}.

\item the symbols of chemical elements should be typed in roman, while its atomic or mass numbers, as well 
as its usage in chemical formulae, require subscripts or superscripts which can be typed only using the math mode in \LaTeX; one way to do this mix is demonstrated in these examples: $^{234}$U,  $^{234}$U$_{92}$, and H$_2$O,

\item similarly to the previous item, physical units should appear in roman spaced away from the number of a quantity, as shown here: $\tau_A=1$ s, $T=273$ K, and if you are interested in nanoscience might find useful that length $L=1$ nm is the same as $L=10$ \AA{}.

\item one of the complicated equations of Project 1, which was often typed as, e.g.,
\begin{equation}
\frac{\frac{1}{\tau_A}N_{B0} e^{-\frac{t}{\tau_B}+\frac{t}{\tau_B}}}{\frac{1}{\tau_A}-\frac{1}{\tau_B}},
\end{equation}
would look much more pleasing to an eye if it had appeared as
\begin{equation}
\frac{N_{B0} e^{-t/\tau_B + t/\tau_B}}{1 - \tau_A/\tau_B} \hspace{0.5in} {\rm or} \hspace{0.5in} \frac{N_{B0}}{1 - \tau_A/\tau_B}  \exp \left(-\frac{t}{\tau_B} + \frac{t}{\tau_B} \right),
\end{equation}

\item take of look of the source of Eq.~(\ref{eq:exp_function}) which contains a  label that can be 
called anywhere in the text, as demonstrated in the source of this item (the best practice is to label 
each equation when it is introduced into the text); the same method applies to Figures where we 
can reference Fig.~\ref{fig:decay} by labeling the set of commands (in the source file) which allow us to input an  EPS file into the \LaTeX document,

\vspace{0.2in}

%
\begin{figure}[ht]
%\centerline{\includegraphics[scale=0.3,angle=0]{decay.eps}}
\caption{Number of radioactive nuclei as a function of time obtained from the Euler method. The time constant for the decay is $\tau=1$ s and the discretization time step is $\Delta t=0.01$ s. }\label{fig:decay}
\end{figure}
%


\item any figure containing plots of physical quantities must have clearly labeled axes, together with units in which these quantities are measured; also, figures without caption are hard to understand---many figures that we are striving to obtain through our {\bf \em computational projects} are attempts to reproduce those in the textbook or research papers, so you can get clues for possible captions (into which you can also input equations explaining the values of relevant parameters) by checking figures of particular chapter related to the project; an example of possible Figure captions for Project 1 is showing in Fig.~\ref{fig:decay},

\item here is an advanced example of typing a symmetric (${\bf A}^{\rm T}={\bf A}$) $2 \times 2$ matrix, which you will find useful for Project 3
\begin{equation} \label{eq:matrix}
{\bf A}=\left( \begin{array}{cc}
     a & b \\
    -b & c
  \end{array} \right).
\end{equation}

\item to type quantum-mechanical formulae in Project 1, you will need wave function in one-dimension $\Psi(x)=\langle x | \Psi \rangle$, wave function in three dimensions $\Psi({\bf r})=\langle {\bf r} | \Psi \rangle$ [or $\Psi(\vec{r})$], and the time-dependent Schr\"{o}dinger equation 
\begin{equation}\label{eq:schrodinger}
i \hbar \frac{\partial \Psi(x,t)}{\partial t} = \left[ -\frac{\hbar^2}{2m}  \frac{\partial^2}{\partial x^2}+V(x) \right] \Psi(x,t),
\end{equation}
where $V(x)$ is some potential in 1D, $\hbar=h/2\pi$ is the Planck constant divided by $2\pi$, and this equation also shows how to type first and higher order partial derivatives of a function,

\item some other useful QM expressions include operators, such as the Hamiltonian $\hat{H}$, and the tensor products of operators, such as the product of Pauli spin operators $\hat{\sigma}_x \otimes \hat{\sigma}_y$; note that in physics research literature and textbooks we rarely use any explicit multiplication symbol, that is instead of $i * \hbar$ we see simply $i\hbar$ in Eq.~(\ref{eq:schrodinger}); some exceptions (due to beauty requirements) are, e.g., $5 \cdot 10^{-9}$ or $5 \times 10^{-9}$, where symbol $\times$ is also employed to multiply different pieces of a long equation which has to be split over several line.

\end{itemize}



\begin{thebibliography}{99}

%Intro

\bibitem{lamport} L.~Lamport, {\em \LaTeX: A Document Preparation System User's Guide and Reference Manual} (Adison-Wesley, Reading, 1994).

\bibitem{giordano} N.~J.~Giordano and H.~Nakanishi, {\em Computational Physics} (Pearson Prentice Hall, Upper Saddle River, 2005).

\bibitem{whelan} N.~D.~Whelan, D.~A.~Goodings, and J.~K.~Cannizzo, Phys. Rev. A {\bf 42}, 742 (1990).

\end{thebibliography}


\end{document}
